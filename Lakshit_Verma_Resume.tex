\documentclass{resume} % Use the custom resume.cls style
\usepackage[hidelinks]{hyperref}
\usepackage{lipsum}
\usepackage{graphicx}
\usepackage{adjustbox}
\usepackage{subcaption}
\usepackage{setspace}
\usepackage{fontawesome}
\usepackage[inline]{enumitem}
\usepackage{xcolor}
\date{\today}
\makeatletter
% This command ignores the optional argument for itemize and enumerate lists
\newcommand{\inlineitem}[1][]{%
\ifnum\enit@type=\tw@
    {\descriptionlabel{#1}}
  \hspace{\labelsep}%
\else
  \ifnum\enit@type=\z@
       \refstepcounter{\@listctr}\fi
    \quad\@itemlabel\hspace{\labelsep}%
\fi}
\makeatother
\parindent=0pt
\usepackage[left=0.3in,top=0.2in,right=0.3in,bottom=0.2in]{geometry} % Document margins
\hyphenpenalty=10000


\newcommand{\tab}[1]{\hspace{.\textwidth}\rlap{#1}}
\newcommand{\itab}[1]{\hspace{0em}\rlap{#1}}

\name{Lakshit Verma}
\address {\faPhone \href{tel:+918448058867}{~(+91) 8448058867} \\
\faEnvelopeO \href{mailto:vermalucky2004@gmail.com}{ vermalucky2004(at)gmail.com} \\
{\faGithub \href{https://github.com/vee1e}{ \underline{GitHub}}} \\
{\faGlobe \href{https://lverma.com/}{ \underline{Portfolio}}} \\
{\faLinkedin \href{https://www.linkedin.com/in/lakshitverma/}{ \underline{Linkedin}}}}


\begin{document}
\begin{rSection}{Skills}

\begin{tabular}{ @{} >{\bfseries}l @{\hspace{8ex}} l }
Languages \ & Python, C/C++, JavaScript, TypeScript, Bash, HTML/CSS, \LaTeX, SQL \\
Frameworks \& Libraries \ & Angular, FastAPI, Material UI, React, Next.js, PyQt, Node.js, Express.js, Tailwind CSS \\
Technologies \ & MongoDB, REST APIs, JSON, XLSForm, Git, Docker, Make/CMake \\
Cybersecurity \ & IDA Pro, Ghidra, Wireshark, Volatility, GDB, EnCase, Sleuthkit, FTK Imager, Btrfs \\
Relevant Coursework \ & Computer Architecture, FPGA Design, Verilog, Microcontrollers, Embedded Systems \\
Management \& Soft Skills \ & Leadership, Technical Management, Problem Solving, Communication, Teamwork, Training \\
\end{tabular}
\end{rSection}

\begin{rSection}{Technical Experience}

\begin{rSubsection}{\bf Student Software Developer, Dhwani RIS}{\em May 2025 -- Present}{\href{https://codeforgovtech.in/dedicated_mentoring_program/}{Code for GovTech DMP}}{Remote}
\item Designed and implemented a FastAPI-based backend parser from scratch enabling bulk conversion of Excel-based survey forms through XLSForm to mForm-compatible JSON, supporting forms with up to 400 questions per upload.
\item Achieved batch processing speeds of 2.16 seconds for 9 forms ($\approx$3,600 questions total), with per-question conversion latency averaging 0.15 ms and single-form save times of 240ms.
\item Built an Angular frontend from scratch for real-time Excel uploads, live previews, inline error reporting, and batch operations, integrated with REST APIs and MongoDB for seamless form lifecycle management.
\item Conducted comprehensive manual and automated QA, validating JSON output against mForm templates and implementing edge-case handling to guarantee reliability and future maintainability.
\item Links: \faGithub~\href{https://github.com/vee1e/bulk-questionnaire-upload}{\textbf{\underline{Source Code}}}, \href{https://github.com/dhwani-ris/bulk-questionnaire-upload/issues/1}{\textbf{\underline{GitHub Issue}}}
\end{rSubsection}

\begin{rSubsection}{\bf Team Member}{\em January 2024 -- Present}{Cryptonite --- Major Student Project}{Manipal, Karnataka, India $\cdot$ On-site}
\item Extensively participated in \href{https://ctftime.org/team/62713/}{150+ Capture The Flag (CTF)} competitions, namely in Forensics and Reverse Engineering.
\item Team achieved \#2 national ranking in 2025 and \#3 national ranking in 2024 on ctftime.org, competing against 10,000+ teams worldwide in prestigious competitions like BSides Bangalore CTF, IIT BHU's Kashi CTF, and NahamCon CTF.
\item Designed and developed 2 original CTF challenges for \href{https://github.com/Cryptonite-MIT/niteCTF-2024/}{niteCTF 2024}: a Minecraft protocol analysis challenge requiring packet parsing and 3D coordinate extraction and an 8-bit VM challenge with NOR gate-based XOR encryption that tested participants' skills in virtual machine analysis and logic gate reverse engineering.
\item Lead 20+ Junior Members in the development and execution of the team's OASIS CTF, an intra-college entry-level cybersecurity competition that attracted 499 participants across 218 teams, including challenge creation, infrastructure setup, and real-time technical support during the 24-hour event.
\end{rSubsection}

\begin{rSubsection}{\bf Core Committee Member}{\em November 2023 -- October 2024}{Manipal University ACM Chapter --- Student Club}{Manipal, Karnataka, India $\cdot$ Hybrid}
\item Co-organized the ``Classified'' ML workshop with 100+ participants, handling dataset curation and technical support.
\item Curated datasets for Epoch 2024 ML competition with 100+ entries, ensuring data quality across multiple domains.
\end{rSubsection}

\end{rSection}

\begin{rSection}{Achievements}
\begin{list}{$\bullet$}{\leftmargin=1em \itemindent=0em}
\itemsep -0.5em
\item Awarded \textbf{\#1 Position} and \textbf{1,00,000 INR} in the \href{https://www.sih.gov.in/sih2024/sih2024-grand-finale-result}{Smart India Hackathon 2024}, under PS 1749.
\item Awarded \textbf{\#1 Position} and \textbf{25,000 INR} in the GITxIITB CTF by KLS GIT, Belagavi \& IIT Bombay Trust Lab.
\item Awarded \textbf{\#1 Position} and \textbf{15,000 INR} in the KJSSE CTF by KJ Somaiya College of Engineering, Mumbai.
\item Awarded \textbf{\#2 Position} and \textbf{10,000 INR} in the SoftLaunch Hackathon by the MIT Innovation Centre.
\item Finalist of Cython 2024, a national-level hackathon organized by FITT, IIT Delhi.
\end{list}
\end{rSection}

\begin{rSection}{Projects}

\begin{rSubsection}{\bf NiteWatch~|~Codebase Proprietary}{\em Python, PyQT, Btrfs, XFS, EnCase, DFIR, Blue Teaming}{}{}
\item Parses XFS and Btrfs file systems from scratch using no external libraries for full recovery of deleted data and metadata.
\item Contains an intuitive PyQt frontend for easy visual analysis and user-friendly file system exploration.
\item Integrates AI analysis using OpenAI's APIs for DFIR-relevant insights and automated threat detection.
\end{rSubsection}

\begin{rSubsection}{\bf YatraGPT~|~\href{https://github.com/vee1e/finova}{\faGithub~\underline{Source Code}}}{\em React, Gemini API, Typescript, NextJS}{}{}
\item An AI travel assistant using Gemini API for trip planning with real-time recommendations.
\item Integrated IRCTC API for train ticket search and implemented backtracking algorithm for optimal route finding.
\item Deployed on Vercel with TypeScript and Next.js for scalable performance and modern web standards.
\end{rSubsection}

\end{rSection}

\begin{rSection}{Education}
    {\bf Manipal Institute of Technology, Manipal, Karnataka} \hfill \textit{July 2023 -- August 2027}  \\
    {Bachelor of Technology --- Electronics Engineering (VLSI Design \& Technology)} \hfill Currently in Vth Semester \\
\end{rSection}

\vspace{-0.8em}
\small \begin{center}This resume was last updated on \DTMnow.\\\end{center}

\end{document}
