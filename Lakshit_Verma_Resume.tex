\documentclass{resume} % Use the custom resume.cls style
\usepackage[hidelinks]{hyperref}
\usepackage{lipsum}
\usepackage{graphicx}
\usepackage{adjustbox}
\usepackage{subcaption}
\usepackage{setspace}
\usepackage{fontawesome}
\usepackage[inline]{enumitem}
\usepackage{xcolor}
\date{\today}
\makeatletter
% This command ignores the optional argument for itemize and enumerate lists
\newcommand{\inlineitem}[1][]{%
\ifnum\enit@type=\tw@
    {\descriptionlabel{#1}}
  \hspace{\labelsep}%
\else
  \ifnum\enit@type=\z@
       \refstepcounter{\@listctr}\fi
    \quad\@itemlabel\hspace{\labelsep}%
\fi}
\makeatother
\parindent=0pt
\usepackage[left=0.3in,top=0.2in,right=0.3in,bottom=0.2in]{geometry} % Document margins
\hyphenpenalty=10000


\newcommand{\tab}[1]{\hspace{.\textwidth}\rlap{#1}}
\newcommand{\itab}[1]{\hspace{0em}\rlap{#1}}

\name{Lakshit Verma}
\address {\faPhone \href{tel:+918448058867}{~(+91) 8448058867} \\
\faEnvelopeO \href{mailto:vermalucky2004@gmail.com}{ vermalucky2004(at)gmail.com} \\
{\faGithub \href{https://github.com/vee1e}{ \underline{GitHub}}} \\
{\faGlobe \href{https://lverma.com/}{ \underline{Portfolio}}} \\
{\faLinkedin \href{https://www.linkedin.com/in/lakshitverma/}{ \underline{Linkedin}}}}


\begin{document}
\vspace{-0.5em}
\begin{rSection}{Technical Experience}

\begin{rSubsection}{\bf Undergraduate Researcher}{\em September 2025 -- Present}{Dr. T.M.A. Pai Endownment Chair}{Manipal, Karnataka, India $\cdot$ Hybrid}
\item Working under Senior Professor Dr. Manohara Pai and Schneider Electric on AI-enabled security analysis and test case generation.
\item Built a novel framework via LLMs to automatically parse CVE, CWE and MITRE ATT\&CK databases across 335K+ records.
\item Created a threat analysis engine through self-healing pipelines combining LLM inference with STRIDE keyword matching.
\item Automated binary vulnerability analysis using Radare with exploit generation and sandboxed verification in controlled environments.
\end{rSubsection}

\begin{rSubsection}{\bf Student Software Developer, Dhwani RIS}{\em May 2025 -- August 2025}{\href{https://codeforgovtech.in/dedicated_mentoring_program/}{Code for GovTech DMP}}{Remote}
\item Developed a FastAPI-based backend parser to enable bulk conversion of Excel-based surveys to mForm compatible JSON files.
\item Built an Angular frontend from scratch for real-time async Excel uploads, live previews, inline error reporting, and batch operations
\item Integrated frontend with REST APIs and MongoDB for seamless form lifecycle management.
\item Achieved batch processing speeds of 2.16 seconds for 9 forms, supporting forms with up to 400+ questions per upload
\item Performed with per-question conversion latency averaging 0.15 ms and single-form save times of 240ms.
\item Created comprehensive testing and CI/CD pipelines, validating JSON and implementing error handling in frontend and backend.
\end{rSubsection}

\begin{rSubsection}{\bf Digital Forensics Head}{\em January 2024 -- Present}{Team Cryptonite --- Major Student Project}{Manipal, Karnataka, India $\cdot$ On-site}
\item Participated in \href{https://ctftime.org/team/62713/}{200+ Capture The Flag (CTF)} competitions, achieving \#3 national ranking in 2024 and 2025 on ctftime.org.
\item Designed and developed 5+ challenges for \href{https://github.com/Cryptonite-MIT/niteCTF-2024/}{niteCTF 2024} and \href{https://github.com/Cryptonite-MIT/niteCTF-2025/}{2025}, focusing on real-world digital forensics scenarios.
\item Trained 11 Junior Members in digital forensics techniques, tools, and methodologies, enhancing the team's overall skill set.
\item Created the ``\href{https://github.com/cryptonite-mit/dfir-gita}{DFIR Gita}'' repository, a comprehensive training resource for future team members in digital forensics.
\end{rSubsection}

\end{rSection}

\begin{rSection}{Achievements}
\begin{list}{$\bullet$}{\leftmargin=1em \itemindent=0em}
\itemsep -0.5em
\item Awarded \textbf{\#1 Position} and \textbf{2,00,000 INR} in the ISEA-ISAP 2026 Hackathon organized by IIT Madras.
\item Awarded \textbf{\#1 Position} and \textbf{1,00,000 INR} in the Smart India Hackathon 2024, held at IIT Jammu.
\item Awarded \textbf{\#1 Position} and \textbf{25,000 INR} in the GITxIITB CTF by KLS GIT, Belagavi \& IIT Bombay Trust Lab.
\item Awarded \textbf{\#1 Position} and \textbf{15,000 INR} in the KJSSE CTF by KJ Somaiya College of Engineering, Mumbai.
\item Awarded \textbf{\#2 Position} and \textbf{10,000 INR} in the SoftLaunch Hackathon by the MAHE Innovation Centre.
\end{list}
\end{rSection}

\begin{rSection}{Projects}

\begin{rSubsection}{\bf NiteWatch}{\em Python, PyQT, Btrfs, DFIR, Operating Systems}{}{}
\item A DFIR application built for Btrfs and XFS filesystems, capable of restoring deleted files along with their complete metadata.
\item Parses filesystem data structures including B+ trees, inode records and superblocks, recovering both deleted and active files.
\item Provides a PyQt-based graphical interface to efficiently navigate, visualize, and interact with reconstructed file system structures.
\end{rSubsection}

\begin{rSubsection}{\bf MFT Parser~|~\href{https://github.com/vee1e/MFT-Parser}{\faGithub~\underline{Source Code}}}{\em C++, Cmake}{}{}
\item A high-performance, cross-platform C++ NTFS forensic parser and CLI for MFT and other core system artifacts.
\item Implements full metadata extraction and deleted file recovery (resident data, non-resident cluster mapping) for forensic workflows.
\item Uses a portable CMake-based build \& test system with recursive artifact detection for analysis on macOS, Linux, and Windows.
\end{rSubsection}

\begin{rSubsection}{\bf Astraeus~|~\href{https://github.com/Addy-Da-Baddy/Astraeus}{\faGithub~\underline{Source Code}}}{\em Python, PyTorch, Three.js, XGBoost}{}{}
\item An ensemble ML application pipeline combined with LSTM/GRU networks for high-accuracy trajectory and collision assessment.
\item Provides real-time ingestion and preprocessing of Celestrak TLE data with feature engineering on orbital elements.
\item Implements and end-to-end system delivering 100ms inference latency, continuous risk monitoring, and 3D viz. of orbital dynamics.
\end{rSubsection}

\begin{rSubsection}{\bf tens~|~\href{https://github.com/vee1e/tens}{\faGithub~\underline{Source Code}}}{\em Objective-C, C, AppKit, Cocoa}{}{}
\item A native macOS port of a plaintext based presentation tool using Cocoa/AppKit, eliminating XQuartz and X11 dependencies.
\item Implements plaintext-based slide rendering with automatic text scaling and native image support using Core Graphics.
\end{rSubsection}

\end{rSection}

\begin{rSection}{Education}
    {\bf Manipal Institute of Technology, Manipal, Karnataka} \hfill \textit{July 2023 -- August 2027}  \\
    {Bachelor of Technology --- Electrical Engineering {\small \textit{(Minor in Computing)}}} \hfill \textit{Currently in VIth Semester} \\
\end{rSection}

\vspace{-3.1em}
\begin{rSection}{}

\begin{tabular}{ @{} >{\bfseries}l @{\hspace{8ex}} l }
Languages \ & Python, C/C++, JavaScript, TypeScript, Bash, HTML/CSS, \LaTeX, SQL \\
Frameworks \& Libraries \ & Angular, FastAPI, Material UI, React, Next.js, PyQt, Node.js, Express.js, Tailwind CSS \\
Technologies \ & MongoDB, REST APIs, JSON, XLSForm, Git, Docker, Make/CMake \\
Cybersecurity \ & IDA Pro, Ghidra, Wireshark, Volatility, GDB, EnCase, Sleuthkit, FTK Imager, Btrfs \\
Management \& Soft Skills \ & Leadership, Technical Management, Problem Solving, Communication, Teamwork, Training \\
\end{tabular}
\end{rSection}

\vfill
\vspace{-0.2em}
{\small \begin{center}This resume was last updated on \DTMnow.\end{center}}
\vspace{-0.5em}

\end{document}
